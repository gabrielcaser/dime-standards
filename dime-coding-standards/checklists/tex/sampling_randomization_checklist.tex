\documentclass{tufte-handout}

\usepackage{librecaslon}


\usepackage{fancyhdr}
\usepackage{hyperref}
\usepackage{tcolorbox} % needed for the text boxes
\usepackage{xcolor}
\usepackage{setspace}
\usepackage{graphics}
\hypersetup{
	colorlinks=true,
	linkcolor=blue,
	urlcolor=cyan,
}
% Set header and footer
\pagestyle{fancy}
\fancyhf{}
\lfoot{\includegraphics[height=1cm,keepaspectratio]{../../img/i2i}}
\cfoot{\includegraphics[height=1cm,keepaspectratio]{../../img/wb}}
\rfoot{\includegraphics[height=1cm,keepaspectratio]{../../img/analytics}}

% Put checkbox to the left of the text
\def\LayoutCheckField#1#2{% label, field
	#2 #1%
}

% Line spacing
\onehalfspacing

% Define DIME Analytics visual identity colors
\definecolor{fontcolor}{HTML}{7A0569}

\titleformat{\section}%
{\Large\rmfamily\bf\color{fontcolor}}% format applied to label+text
{\llap{\colorbox{fontcolor}{\parbox{1.5cm}{\hfill\huge\color{fontcolor}\thesection}}}}% label
{2pt}% horizontal separation between label and title body
{}% before the title body
[]% after the title body

\titleformat{\subsection}%
{\large\rmfamily\color{fontcolor}}% format applied to label+text
{}% label
{1.5pt}% horizontal separation between label and title body
{}% before the title body
[]% after the title body

\newcommand{\dimeCheckBox}[1]{\CheckBox[height=0.01cm, width=0.4cm, bordercolor=gray]{#1}}
\newcommand{\dimeTextField}[3]{\TextField[name=#1, height=0.3cm, width=#2, bordercolor=gray]{#3}}


\newcommand{\titleBox}[1]{
	\begin{tcolorbox}
		[colframe = fontcolor,
		colback = fontcolor,
		sharp corners,
		halign = flush center,
		valign = center,
		height = 0.3\textwidth,
		after skip = 1cm]
		#1
	\end{tcolorbox}
}

\begin{document}
	\begin{fullwidth}
		
		\titleBox{
			\textcolor{white}{\LARGE{\textbf{DIME Analytics \\ Peer Code Review - Sampling and Randomization}} \\
				\Large\textbf{{v1.0}}}
		}

  \dimeTextField{time}{2.5 cm}{Reviewer:} 

        \dimeTextField{time}{2.5 cm} {Coder:}
		
		\textbf{NOTE:} Make sure that fill out this checklist \textbf{ONLY IF} your partner's submission includes    \textbf{sampling and/or randomization} tasks. 
		
		\section*{Sampling and Randomization tasks}
  This checklist lists important factors to consider while reviewing your code review partner's \textbf{sampling/randomization} code. Please fill this checklist, and submit it as an attachment when you submit \href{https://survey.wb.surveycto.com/collect/code_review_summary?caseid=}{this detailed form.}
  \hfill \break 
  \break
  \textbf{Note:} We are piloting this checklist as part of improving the code review process. The questions in this list are therefore \textbf{not required}, and we welcome your feedback on the points listed below.

  
		\begin{itemize}
			\setlength\itemsep{-0.1em}
   
   
    \item[]{\textbf{Sampling:}
   
            \begin{itemize}
            
            \item[]\dimeCheckBox{\textbf{The code clearly identifies the population eligible for sampling}. (This should be in the form of an original dataset that is fixed, unless sampling is continuous or ongoing due to research design; in that case, this should be clearly described)}
         
            
            \begin{itemize}
                \item[]\dimeCheckBox{The population dataset is uniquely identified - and the code tests for it.}
                
                \item[]\dimeCheckBox{The population dataset contains all variables used in the sampling process, such as clusters and strata.}
                
                \item[]\dimeCheckBox{\textbf{Advanced check:} Clusters, strata, and individual-level IDs do not contain unmasked information about the identity of the clusters, strata, or individuals.} 

                \item[] \dimeCheckBox{\textbf{Advanced check:} The original dataset should be checks for stability when loaded, using some combination of commands like \texttt{datasignature}, file hashing, \texttt{assert}, and/or \texttt{isid, sort} (or an equivalent in other coding languages.)}
                
            \end{itemize}
            
            
			\item[] \dimeCheckBox{\textbf{The method and rationale for random processes is described in documentation and understandable in code}, including:} 
            
            \begin{itemize}
                
                \item[]\dimeCheckBox{The overall sampling strategy is clearly documented, such as any characteristics or levels of observation that affect probabilities of inclusion/exclusion from sample.}
                
                \item[] \dimeCheckBox{Handling of strata and clusters are implemented as described in the research design, for example, using the procedures described by \texttt{randtreat} in Stata (or equivalent).}

                \item[] \dimeCheckBox{Handling of unequal strata and cluster sizes is clearly specified, appropriate, and justified.}

                \item[] \dimeCheckBox{Probability weights are calculated and stored for each unit, when required.}
                
            \end{itemize}
            
            \item[] \dimeCheckBox{\textbf{Random processes-generating samples are implemented in a reproducible script}, including:} 
            
            \begin{itemize}
                \item[] \dimeCheckBox{Version settings are explicitly set according to software requirements (in Stata, the \texttt{version} or \texttt{ieboilstart} commands)}
                \item[] \dimeCheckBox{Seed is set using a unique, random seed generated from an external source (such as \href{http://bit.ly/stata-random}{random.org}, for each random process that is intended to be independent.} 
                \item[] \dimeCheckBox{Population dataset is sorted uniquely before each independent random process.}
            \end{itemize}

            \dimeCheckBox{\textbf{The script outputs a resulting dataset of results of the random processes} for each potential sampling group (or treatment arm)} 
            \begin{itemize}
			\item[] \dimeCheckBox{If the sampling result is intended to be “finalized” for field use, there is a logic switch to ensure that final data cannot be overwritten by new data. For example:}
            \begin{verbatim}         
            if `replace_results` == 1 save `results` , replace
            else di as err "Results not saved: To replace results, toggle logic switch.
            \end{verbatim}
                \item[] \dimeCheckBox{The output dataset is created with corresponding codebooks, value labels, and variable labels describing all results of the random processes.}
                \item[] \dimeCheckBox{The final sampling results are stable and reproducible}
            \end{itemize}         
		\end{itemize}

        \hfill 
        
        \item[]{\textbf{Random treatment assignment:}}

            \begin{itemize}
            
            \item[]\dimeCheckBox{The project uses a script to randomize treatment assignment.}
            
            \begin{itemize}
                \item[]\dimeCheckBox{If not, the reason is fully documented.}
            \end{itemize}

             \item[]\dimeCheckBox{The code sets the version of Stata before randomizing treatment assignment.}
            
        
            \item[]\dimeCheckBox{No new observations are created in the section where treatment is randomly assigned.}

            \item[]\dimeCheckBox{The data is sorted before randomizing treatment assignment.}
            
            \begin{itemize}
                \item[]\dimeCheckBox{If yes, the data is sorted on a variable that is uniquely identifying.}
            \end{itemize}

            \item[]\dimeCheckBox{The code uses \texttt{set seed}} before the sorting.

            \begin{itemize}
            
                \item[]\dimeCheckBox{If yes, the seed is set using a unique, random seed generated from an external source (such as         \href{http://bit.ly/stata-random}{random.org}, for each random process that is intended to be independent.}
                
            \end{itemize}

            \item[]\dimeCheckBox{The code creates a categorical variable to identify the treatment and control groups.}

            \item[]\dimeCheckBox{The code for randomozing treatment assignment is reproducible and consistent across multiple runs. For more information, you can refer to \href{https://osf.io/ys3fv/}{this resource on iedorep}.}

        \end{itemize}
        
  \end{itemize}


		
		\vspace{.6em}
       
		
	\end{fullwidth}
\end{document}

